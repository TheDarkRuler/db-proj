\documentclass[a4paper,12pt]{report}

\usepackage{alltt, fancyvrb, url}
\usepackage{graphicx}
\usepackage[utf8]{inputenc}
\usepackage{float}
\usepackage{hyperref}

\usepackage[italian]{babel}

\usepackage[italian]{cleveref}

\title{\textbf{Elaborato per il corso di Basi Di Dati \\ A.A 2022/2023}}

\author{Ettore Farinelli \\ ettore.fainelli@studio.unibo.it \\ 0001019995}

\date{\today}

\begin{document}

\maketitle

\tableofcontents

\chapter{Analisi dei requisiti}
Si pone l'obbiettivo di realizzare un database capace di gestire un organizzazione di arti marziali come può essere, 
per esempio, \textsc{L'Ultimate Fighting Championship, (UFC)}. La base di dati dovrà quindi essere capace di registrare 
nuovi \textbf{lottatori} e in caso rimuoverli(squalifica, infortuneo, ritiro). Inoltre sarà possibile registrare \textbf{eventi}, 
dove i combattenti si scontreranno aggiornando (dopo gli \textbf{scontri}), gli \textbf{score} dei partecipanti 
e in caso le \textbf{classifiche}.

\section{Intervista}
A seguito di una prima intervista si sono ottenute le seguenti richieste:\medskip

Per ogni partecipante alla lega bisogna tenere traccia del nome, cognome, codice fiscale, team, data di nascita, peso e \textbf{arte 
marziale} in cui vuole lottare (più possibilità disponibili). Alla iscrizione di un nuovo lottatore esso verrà inserito all'ultimo 
posto nella classifica della propria categoria, e dovrà effettuare almeno un combattimento all'anno, oppure verrà automaticamente 
squalificato dalla lega. Saranno presenti diverse classifiche per ogni tipo di categoria dove i lottatori saranno ordinati 
in base ai loro \textbf{record} (V, P, S), dove le vittorie assegnano 3 punti, i pareggi 1 e le sconfitte 0.\par
Le categorie in cui verranno suddivisi i membri della lega sono: \textit{Peso Piuma} (fino a 65kg), \textit{Welterweight} 
(65kg - 77kg), \textit{Peso Medio} (77kg - 84kg) e \textit{Pesi Massimi} (da 84kg in poi). Inoltre ci sarà una divisione in 
arti marziali: \textit{MMA} (Mixed Martial Arts), \textit{BJJ} (Brazial Jiu Jitsu) e infine \textit{Muay Thai}. Per ogni 
partecipante dovranno essere inserite le discipline di competenza possibilmente modificabili in futuro (sarà gestito in maniera 
analoga il peso). Inoltre, partecipanti appartenenti a una determinata categoria potranno scontrarsi solo con altri membri 
della stessa. Gli eventi saranno constituiti da almeno 2 combattimenti ciascuno, bisognerà tener traccia del: nome dello stadio, 
luogo (indirizzo), costo noleggio stadio, spesa staff, data, orario inizio, orario fine, biglietti standard venduti, biglietti 
premium venduti, introiti netti. Ogni lottatore partecipante riceverà un pagamento extra e verrà calcolata una quantità di guadagni 
tramite pubblicità, tutto in base al numero di biglietti venduti per l'evento, così da poter calcolare gli introiti dell'evento, 
il quale verrà aggiunto in una \textsc{History} dove si potranno visualizzare tutti gli eventi passati, 
avendo infine la possibilità di visualizzare i guadagni e le spese totali della lega.

\section{Definizioni}
\begin{itemize}
    \item \textbf{Lottatore}: partecipante alla lega.
    \item \textbf{Organizzatore}: amministratore che ha l'accesso al database e le autorizzazioni per gestirlo. 
    \item \textbf{Evento}: un insieme di scontri avente un luogo e una data.
    \item \textbf{Scontro}: un incontro tra due lottatori della stessa categoria.
    \item \textbf{Classifica}: lista numerata in ordine dal partecipante migliore al peggiore in base ai record personali.
    \item \textbf{Arte marziale}: disciplina frequentante da un partecipante.
    \item \textbf{Record}: terna di vittorie, pareggi, sconfitte (V, P, S), ogni prtecipante ha la sua che definisce la sua posizione in 
    classifica.
\end{itemize}

\subsection{Operazioni Amministratore}
\begin{itemize}
    \item Registrare un nuovo lottatore.
    \item Rimuovere un lottatore.
    \item Organizzare uno evento.
    \item Visualizzare le classifiche.
    \item Modificare i dati dei partecipanti.
\end{itemize}
\end{document}
