\documentclass[a4paper,12pt]{report}

\usepackage{alltt, fancyvrb, url}
\usepackage{graphicx}
\usepackage[utf8]{inputenc}
\usepackage{float}
\usepackage{hyperref}

\usepackage[italian]{babel}

\usepackage[italian]{cleveref}

\title{\textbf{Elaborato per il corso di Basi Di Dati \\ A.A 2022/2023}}

\author{Ettore Farinelli \\ ettore.fainelli@studio.unibo.it \\ 0001019995}

\date{\today}

\begin{document}

\maketitle

\tableofcontents

\chapter{Analisi dei requisiti}
Si pone l'obbiettivo di realizzare un database capace di gestire un organizzazione di arti marziali come può essere, 
per esempio, \textsc{L'Ultimate Fighting Championship, (UFC)}. La base di dati dovrà quindi essere capace di registrare 
nuovi lottatori e in caso rimuoverli(squalifica, infortuneo, ritiro). Inoltre si potrà realizzare un evento, dove 
i combattenti si scontreranno aggiornando (dopo gli scontri), gli score dei partecipanti e in caso le classifiche.

\section{Intervista}
A seguito di una prima intervista si sono ottenute le seguenti richieste:\medskip

Per ogni partecipante alla lega bisogna tenere traccia del nome, cognome, codice fiscale, team, data di nascita, peso e arte 
marziale in cui vuole lottare (più possibilità disponibili). Alla iscrizione di un nuovo lottatore esso verrà inserito in fondo 
alla classifica della propria categoria, e dovrà effettuare almeno un combattimento all'anno, oppure verrà automaticamente 
squalificato dalla lega. Saranno presenti diverse classifiche per ogni tipo di categoria dove i lottatori saranno ordinati 
in base ai loro record (V, S, P), dove le vittorie assegnano 3 punti, i pareggi 1 e le sconfitte 0.


\end{document}